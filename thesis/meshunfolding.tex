% Copyright (C) 2014-2019 by Thomas Auzinger <thomas@auzinger.name>

\documentclass[draft,final]{vutinfth} % Remove option 'final' to obtain debug information.

% Load packages to allow in- and output of non-ASCII characters.
\usepackage{lmodern}        % Use an extension of the original Computer Modern font to minimize the use of bitmapped letters.
\usepackage[T1]{fontenc}    % Determines font encoding of the output. Font packages have to be included before this line.
\usepackage[utf8]{inputenc} % Determines encoding of the input. All input files have to use UTF8 encoding.

% Extended LaTeX functionality is enables by including packages with \usepackage{...}.
\usepackage{amsmath}    % Extended typesetting of mathematical expression.
\usepackage{amssymb}    % Provides a multitude of mathematical symbols.
\usepackage{mathtools}  % Further extensions of mathematical typesetting.
\usepackage{microtype}  % Small-scale typographic enhancements.
\usepackage[inline]{enumitem} % User control over the layout of lists (itemize, enumerate, description).
\usepackage{multirow}   % Allows table elements to span several rows.
\usepackage{booktabs}   % Improves the typesettings of tables.
\usepackage{subcaption} % Allows the use of subfigures and enables their referencing.
\usepackage[ruled,linesnumbered,algochapter]{algorithm2e} % Enables the writing of pseudo code.
\usepackage[usenames,dvipsnames,table]{xcolor} % Allows the definition and use of colors. This package has to be included before tikz.
\usepackage{nag}       % Issues warnings when best practices in writing LaTeX documents are violated.
\usepackage{todonotes} % Provides tooltip-like todo notes.
\usepackage{hyperref}  % Enables cross linking in the electronic document version. This package has to be included second to last.
\usepackage[acronym,toc]{glossaries} % Enables the generation of glossaries and lists fo acronyms. This package has to be included last.

% Define convenience functions to use the author name and the thesis title in the PDF document properties.
\newcommand{\authorname}{Thorsten Korpitsch} % The author name without titles.
\newcommand{\thesistitle}{3D Mesh Unfolding} % The title of the thesis. The English version should be used, if it exists.

% Set PDF document properties
\hypersetup{
    pdfpagelayout   = TwoPageRight,           % How the document is shown in PDF viewers (optional).
    linkbordercolor = {Melon},                % The color of the borders of boxes around crosslinks (optional).
    pdfauthor       = {\authorname},          % The author's name in the document properties (optional).
    pdftitle        = {\thesistitle},         % The document's title in the document properties (optional).
    pdfsubject      = {Subject},              % The document's subject in the document properties (optional).
    pdfkeywords     = {a, list, of, keywords} % The document's keywords in the document properties (optional).
}

\setpnumwidth{2.5em}        % Avoid overfull hboxes in the table of contents (see memoir manual).
\setsecnumdepth{subsection} % Enumerate subsections.

\nonzeroparskip             % Create space between paragraphs (optional).
\setlength{\parindent}{0pt} % Remove paragraph identation (optional).

\makeindex      % Use an optional index.
\makeglossaries % Use an optional glossary.
%\glstocfalse   % Remove the glossaries from the table of contents.

% Set persons with 4 arguments:
%  {title before name}{name}{title after name}{gender}
%  where both titles are optional (i.e. can be given as empty brackets {}).
\setauthor{}{\authorname}{}{male}
\setadvisor{Ph.D.}{Hsiang-Yun Wu}{}{female}

% For bachelor and master theses:
%\setfirstassistant{Pretitle}{Forename Surname}{Posttitle}{male}
%\setsecondassistant{Pretitle}{Forename Surname}{Posttitle}{male}
%\setthirdassistant{Pretitle}{Forename Surname}{Posttitle}{male}

% For dissertations:
%\setfirstreviewer{Pretitle}{Forename Surname}{Posttitle}{male}
%\setsecondreviewer{Pretitle}{Forename Surname}{Posttitle}{male}

% For dissertations at the PhD School and optionally for dissertations:
%\setsecondadvisor{Pretitle}{Forename Surname}{Posttitle}{male} % Comment to remove.

% Required data.
\setregnumber{01529243}
\setdate{01}{03}{2019} % Set date with 3 arguments: {day}{month}{year}.
\settitle{\thesistitle}{3D Mesh Unfolding} % Sets English and German version of the title (both can be English or German). If your title contains commas, enclose it with additional curvy brackets (i.e., {{your title}}) or define it as a macro as done with \thesistitle.
\setsubtitle{}{} % Sets English and German version of the subtitle (both can be English or German).

% Select the thesis type: bachelor / master / doctor / phd-school.
% Bachelor:
\setthesis{bachelor}
%
% Master:
%\setthesis{master}
%\setmasterdegree{dipl.} % dipl. / rer.nat. / rer.soc.oec. / master
%
% Doctor:
%\setthesis{doctor}
%\setdoctordegree{rer.soc.oec.}% rer.nat. / techn. / rer.soc.oec.
%
% Doctor at the PhD School
%\setthesis{phd-school} % Deactivate non-English title pages (see below)

% For bachelor and master:
\setcurriculum{Media Informatics and Visual Computing}{Medieninformatik und Visual Computing} % Sets the English and German name of the curriculum.

% For dissertations at the PhD School:
%\setfirstreviewerdata{Affiliation, Country}
%\setsecondreviewerdata{Affiliation, Country}


\begin{document}

\frontmatter % Switches to roman numbering.
% The structure of the thesis has to conform to
%  http://www.informatik.tuwien.ac.at/dekanat

\addtitlepage{naustrian} % German title page (not for dissertations at the PhD School).
\addtitlepage{english} % English title page.
\addstatementpage

\begin{danksagung*}
\todo{Ihr Text hier.}
\end{danksagung*}

\begin{acknowledgements*}
\todo{Enter your text here.}
\end{acknowledgements*}

\begin{kurzfassung}
\todo{Ihr Text hier.}
\end{kurzfassung}

\begin{abstract}
\todo{Enter your text here.}
\end{abstract}

% Select the language of the thesis, e.g., english or naustrian.
\selectlanguage{english}

% Add a table of contents (toc).
\tableofcontents % Starred version, i.e., \tableofcontents*, removes the self-entry.

% Switch to arabic numbering and start the enumeration of chapters in the table of content.
\mainmatter

\chapter{Introduction}

\begin{itemize}
	\item What is 3D-Mesh-Unfolding
	\item Why do 3D-Mesh-Unfolding? Papercraft, Self-folding Materials
	\item What problems arise when unfolding a 3D-Model? - Overlaps, Distortions
\end{itemize}

\section{Motivation}

\begin{itemize}
	\item Adding Gluetags makes the problem more difficult
	\item Previous work mostly focused on unfolding techniques
	\item Gluetags can improve reconstruction experience
	\item Simple Algorithms can already yield good results
\end{itemize}

\section{Goal}

\begin{itemize}
	\item Calculate Gluetags for the 3D Model
	\item Implement algorithm to transform 3D models into planar patch
	\item Unfold a minimum Amount of Gluetags and add them to the planar patch
	\item Detect Overlaps between Faces and Gluetags
	\item Resolve Overlaps using simulated annealing to find a global optimum
	\item Visualize the process of unfolding and enable interaction with 3D Model and planar Patch
	\item Evaluate the performance and limitations of the Algorithm
\end{itemize}

\chapter{Definitions}

\section{Dualgraph}

\begin{itemize}
	\item Dualgraph is calculated from a Mesh that is interpreteda s a graph, where the vertices are nodes and the edges between vertices are edges between nodes
	\item Faces are connected through edges of the Graph, these connections turn into edges of the dual graph
	\item Faces are nodes in the Dual Graph
	\item Edges of the Dualgraph can either be bent or cut for the unfolding, leading to a planar patch that can be reconstructed
	\item Image of a dual graph example
\end{itemize}

\section{Minimum Spanning Tree}

\begin{itemize}
	\item Minimum spanning tree is the minimum amount of edges to connect all nodes, therefore it is acyclic
	\item Can be used to find a possible unfolding
	\item Simple to implement and good performance
	\item Image showing the Spanning tree in an unfolded patch
\end{itemize}

\section{Simulated Annealing}

\begin{itemize}
	\item Simmulated annealing is a process to find an global optimum as opposed to a local optimum in a greedy algorithm
	\item Simple to implement and can yield results within a set range of iterations
	\item Global optimum is needed for an unfolding, since no overlaps can be produced to find a result
\end{itemize}

\chapter{Methodology}

\section{Overview}

\begin{itemize}
	\item Overview of the whole algorithm, which steps are done
	\item Data is read from off files, models need to be complete, without holes
	\item All possible gluetags are calculated for the 3D Model
	\item Possible unfoldings are calculated in each iteration of the annealing process
	\item Each possible unfolding is checked for overlaps and evaluated
\end{itemize}

\section{Data}

\begin{itemize}
	\item meshes need to be triangulated and without doubled vertices, also no holes can be present
	\item for calculation the faces and their connecting edges gets indexed to improve performance as edges are only calculated using indices
\end{itemize}

\section{Dualgraph Calculation}

\begin{itemize}
	\item dual graph can be calculated from the original mesh 
	\item dual graph needs to be only calculate once as it does not change
	\item for each face it's neighbours are calculated from the halfedges and the opposite halfedges
	\item edges are saved with the faces indices for later calculation
\end{itemize}

\section{Gluetag Calculation}

\begin{itemize}
	\item for each edge a gluetag is already calculated in 3D
	\item the vertices for each edge are the base of the gluetag
	\item gluetags are calculated for each edge, to target both faces of the edge both once as the source of the gluetag and the target of it
	\item gluetags have a trapezoid shape with the smaller part directing to the targeted face, they will take up a maximum of 1/5 of the target space
	\item gluetags only need to be calculated once as they don't change
	\item gluetags can in theory change in shape and size, depending on the preferences of the user, whereas the bigger the gluetags are the smaller the solution space for unfoldings will be
\end{itemize}

\section{Minimum Spanning Tree Calculation}

\begin{itemize}
	\item for the spanning tree the edges are sorted by their weight
	\item edges are iterated through and filled into an adjacence list
	\item for each iteration the adjacence list is checked if the added edge results in a cycle, if it did, the edge will be removed from the adjacence list and added to the list of cut edges
	\item fo each iteration a list with discovered faces is used to determine if the graph is not only acyclic but also complete, as one complete unfolding is necessary
	\item the resulting edges are kept in two list, the edges that are kept and folded and the edges that are cut
\end{itemize}

\section{Transformation from 3D to 2D}

\begin{itemize}
	\item the adjacence list calculated in the previous step is also used to determine the order of unfolding the faces
	\item the first face is a special case
	\item say the vertices is made up of A,B,C in 3D and a,b,c in 2D
	\item a can be set to 0,0 disregarding the original position of A
	\item b can be calculated by using the distance of A and B and setting it to 0,dist(A,B)
	\item c is more complex and has 2 solutions ;INSERT FORMULA
	\item for the all other vertices two points are already given
	\item which points are given can be calculated by which Vertices are the same in 3D
	\item the third point needs to be on the opposite side of the edge A1B1 compared to the third point of the previous triangle
	\item gluetags can be treated the same as normal triangles, as they are built out of two triangles, their calculation is only done if the triangles unfold without overlaps to save computational time
\end{itemize}

\section{Overlap Detection}

\begin{itemize}
	\item for overlap detection the unfolded triangles are iterated in two nested loops as the second ones start is increased to avoid double checking for overlaps
	\item overlapping area can be calculated similar to the sunderland clipping algorithm, though a more simpler approach can be used
	\item two example cases of overlaps; IMAGE
	\item all edges are checked if intersections occur, there is no overlap
	\item otherwise the overlap can be defined as a polygon which can be calculated by calculating all points that are within the other triangle and the intersections of the edges
	\item formulas for the calculation of points inside a triangle and intersections of edges
	\item the result of these points defines the overlapping polygon and the area can be calculated
	\item the overlapping area is used to determine the quality of the unfolding
	\item gluetag overlap detection is similar to the triangle overlaps, as the gluetag consist of two triangles and each triangle can be checked for overlaps
\end{itemize}

\section{Simmulated Annealing Process}

\begin{itemize}
	\item simmulated annealing is the process that combines previous explained points
	\item a temperature is defined that will be lowered in each iteration until the minimum temperature is reached
	\item before iterating a starting state is calculated by random, the energy of the state should be minimized in the process, as the energy is the sum of the overlap area
	\item first a random neighborhood state is calculated for the state s
	\item then the faces are unfolded and their overlaps are calculated
	\item if no faces overlap, the gluetags are unfolded and the overlaps are calculated
	\item anyways the energy of the overlaps is calculated	
	\item if it is smaller than the previous, it is used for future calculations, else the old state will be used
	\item but for a small chance ;INSERT FORMULA;, a worse move is taken to lessen the probability of getting stuck in a local minimum
	\item if the energy of the state is zero, an unfolding was found and the loop ends
\end{itemize}

\chapter{Evaluation}

\section{Performance}

\begin{itemize}
	\item TABLE that evaluates the performance for different meshes
	\item pictures of unfoldings vs. original meshes
	\item graph depicting performance vs. mesh size
	\item Performance degrades the bigger the mesh gets
	\item optimal performance is below 200 faces
	\item Over 700 faces solution space gets so small and random walk is unlikely to find an unfolding
\end{itemize}

\section{Limitations}

\begin{itemize}
	\item random walk in annealing can be problematic, especially for big and complex meshes - sometimes finds fast solution, sometimes doesn't, the bigger the mesh the unliklier a global optimum with an energy of 0 can be found
	\item bigger meshes need far more iterations and computation time in each iteration -> more sophisticated algorithm needed to scale better
	\item amount of gluetags cannot be calculated in advance - unfolding decides how many gluetags are needed, same unfolding can have a different amount of gluetags depending on their position, bruteforcing of gluetags takes too long; insert formula and example
\end{itemize}

\chapter{Discussion}

\begin{itemize}
	\item if i get a result: performance vs bruteforcing a solution
	\item formula calculating the selection of edges; search space is really big and solution space is hard to calculate but rather small
	\item even sophisticated algorithms for unfolding without gluetags have performance problems if the mesh gets too big
\end{itemize}

\chapter{Conclusion}

\section{Summary}

\begin{itemize}
	\item ???? overall it works fast for smaller meshes, but has its limitations as discussed before
\end{itemize}

\section{Future Work}

\begin{itemize}
	\item post process gluetags to resolve very small overlap areas to cut down computation time
	\item change minimum spanning tree to more sophisticated calculation to get an unfolding faster than with a random walk
	\item find a way to make search space smaller for gluetags to improve performance
\end{itemize}

\backmatter

% Use an optional list of figures.
\listoffigures % Starred version, i.e., \listoffigures*, removes the toc entry.

% Use an optional list of tables.
\cleardoublepage % Start list of tables on the next empty right hand page.
\listoftables % Starred version, i.e., \listoftables*, removes the toc entry.

% Use an optional list of alogrithms.
\listofalgorithms
\addcontentsline{toc}{chapter}{List of Algorithms}

% Add an index.
\printindex

% Add a glossary.
\printglossaries

% Add a bibliography.
\bibliographystyle{alpha}
\bibliography{meshunfolding}

\end{document}