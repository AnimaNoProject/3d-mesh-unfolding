% Copyright (C) 2014-2019 by Thomas Auzinger <thomas@auzinger.name>

\documentclass[draft,final]{vutinfth} % Remove option 'final' to obtain debug information.

% Load packages to allow in- and output of non-ASCII characters.
\usepackage{lmodern}        % Use an extension of the original Computer Modern font to minimize the use of bitmapped letters.
\usepackage[T1]{fontenc}    % Determines font encoding of the output. Font packages have to be included before this line.
\usepackage[utf8]{inputenc} % Determines encoding of the input. All input files have to use UTF8 encoding.

% Extended LaTeX functionality is enables by including packages with \usepackage{...}.
\usepackage{amsmath}    % Extended typesetting of mathematical expression.
\usepackage{amssymb}    % Provides a multitude of mathematical symbols.
\usepackage{mathtools}  % Further extensions of mathematical typesetting.
\usepackage{microtype}  % Small-scale typographic enhancements.
\usepackage[inline]{enumitem} % User control over the layout of lists (itemize, enumerate, description).
\usepackage{multirow}   % Allows table elements to span several rows.
\usepackage{booktabs}   % Improves the typesettings of tables.
\usepackage{subcaption} % Allows the use of subfigures and enables their referencing.
\usepackage[ruled,linesnumbered,algochapter]{algorithm2e} % Enables the writing of pseudo code.
\usepackage[usenames,dvipsnames,table]{xcolor} % Allows the definition and use of colors. This package has to be included before tikz.
\usepackage{nag}       % Issues warnings when best practices in writing LaTeX documents are violated.
\usepackage{todonotes} % Provides tooltip-like todo notes.
\usepackage{hyperref}  % Enables cross linking in the electronic document version. This package has to be included second to last.
\usepackage[acronym,toc]{glossaries} % Enables the generation of glossaries and lists fo acronyms. This package has to be included last.

% Define convenience functions to use the author name and the thesis title in the PDF document properties.
\newcommand{\authorname}{Thorsten Korpitsch} % The author name without titles.
\newcommand{\thesistitle}{3D Mesh Unfolding} % The title of the thesis. The English version should be used, if it exists.

% Set PDF document properties
\hypersetup{
    pdfpagelayout   = TwoPageRight,           % How the document is shown in PDF viewers (optional).
    linkbordercolor = {Melon},                % The color of the borders of boxes around crosslinks (optional).
    pdfauthor       = {\authorname},          % The author's name in the document properties (optional).
    pdftitle        = {\thesistitle},         % The document's title in the document properties (optional).
    pdfsubject      = {Subject},              % The document's subject in the document properties (optional).
    pdfkeywords     = {a, list, of, keywords} % The document's keywords in the document properties (optional).
}

\setpnumwidth{2.5em}        % Avoid overfull hboxes in the table of contents (see memoir manual).
\setsecnumdepth{subsection} % Enumerate subsections.

\nonzeroparskip             % Create space between paragraphs (optional).
\setlength{\parindent}{0pt} % Remove paragraph identation (optional).

\makeindex      % Use an optional index.
\makeglossaries % Use an optional glossary.
%\glstocfalse   % Remove the glossaries from the table of contents.

% Set persons with 4 arguments:
%  {title before name}{name}{title after name}{gender}
%  where both titles are optional (i.e. can be given as empty brackets {}).
\setauthor{}{\authorname}{}{male}
\setadvisor{Ph.D.}{Hsiang-Yun Wu}{}{female}

% For bachelor and master theses:
%\setfirstassistant{Pretitle}{Forename Surname}{Posttitle}{male}
%\setsecondassistant{Pretitle}{Forename Surname}{Posttitle}{male}
%\setthirdassistant{Pretitle}{Forename Surname}{Posttitle}{male}

% For dissertations:
%\setfirstreviewer{Pretitle}{Forename Surname}{Posttitle}{male}
%\setsecondreviewer{Pretitle}{Forename Surname}{Posttitle}{male}

% For dissertations at the PhD School and optionally for dissertations:
%\setsecondadvisor{Pretitle}{Forename Surname}{Posttitle}{male} % Comment to remove.

% Required data.
\setregnumber{01529243}
\setdate{01}{03}{2019} % Set date with 3 arguments: {day}{month}{year}.
\settitle{\thesistitle}{3D Mesh Unfolding} % Sets English and German version of the title (both can be English or German). If your title contains commas, enclose it with additional curvy brackets (i.e., {{your title}}) or define it as a macro as done with \thesistitle.
\setsubtitle{}{} % Sets English and German version of the subtitle (both can be English or German).

% Select the thesis type: bachelor / master / doctor / phd-school.
% Bachelor:
\setthesis{bachelor}
%
% Master:
%\setthesis{master}
%\setmasterdegree{dipl.} % dipl. / rer.nat. / rer.soc.oec. / master
%
% Doctor:
%\setthesis{doctor}
%\setdoctordegree{rer.soc.oec.}% rer.nat. / techn. / rer.soc.oec.
%
% Doctor at the PhD School
%\setthesis{phd-school} % Deactivate non-English title pages (see below)

% For bachelor and master:
\setcurriculum{Media Informatics and Visual Computing}{Medieninformatik und Visual Computing} % Sets the English and German name of the curriculum.

% For dissertations at the PhD School:
%\setfirstreviewerdata{Affiliation, Country}
%\setsecondreviewerdata{Affiliation, Country}


\begin{document}

\frontmatter % Switches to roman numbering.
% The structure of the thesis has to conform to
%  http://www.informatik.tuwien.ac.at/dekanat

\addtitlepage{naustrian} % German title page (not for dissertations at the PhD School).
\addtitlepage{english} % English title page.
\addstatementpage

\begin{danksagung*}
\todo{Ihr Text hier.}
\end{danksagung*}

\begin{acknowledgements*}
\todo{Enter your text here.}
\end{acknowledgements*}

\begin{kurzfassung}
\todo{Ihr Text hier.}
\end{kurzfassung}

\begin{abstract}
\todo{Enter your text here.}
\end{abstract}

% Select the language of the thesis, e.g., english or naustrian.
\selectlanguage{english}

% Add a table of contents (toc).
\tableofcontents % Starred version, i.e., \tableofcontents*, removes the self-entry.

% Switch to arabic numbering and start the enumeration of chapters in the table of content.
\mainmatter

\chapter{Introduction}

\begin{itemize}
	\item Describe what 3D Meshunfolding is
	\item What can it be used for? - Papercraft mostly, self folding materials
	\item What Problems arise? - Performance, Distortion, Overlaps
\end{itemize}

\section{Motivation}

\begin{itemize}
	\item Can be used for Papercraft and Papercraft is fun
	\item Previous work mostly focused on unfolding
	\item Gluetags can improve reconstruction experience
	\item Can simple algorithms yield good results?
\end{itemize}

\section{Goal}

\begin{itemize}
	\item Implement algorith to transform 3D models into planar patch
	\item add minimum amount of gluetags, to have stable structure
	\item detect overlaps and resolve them
	\item visualize all steps of the algorithm with opengl
	\item evaluate the findings
\end{itemize}

\chapter{Definitions}

\section{Dualgraph}

\begin{itemize}
	\item what is a dual graph
	\item describe what it is used for in this problem
\end{itemize}

\section{Minimum Spanning Tree}

\begin{itemize}
	\item what is a minimum spanning tree
	\item how is it related to this problem
\end{itemize}

\section{Simulated Annealing}

\begin{itemize}
	\item what is simulated annealing
	\item how can it be used for this problem
\end{itemize}

\chapter{Implementation/Methodology}

\section{Overview}

\begin{itemize}
	\item general approach - what steps do i have to do:
	\item read the data from files
	\item display and interact with 3d model
	\item calculation of gluetags
	\item unfolding of model
\end{itemize}

\section{Datastructure}

\begin{itemize}
	\item cgal datastructure as a base - short explaination why polyhedron mesh is usefull -> halfedges linking faces together
	\item own datastructures description planarTriangles vs. 3D Triangles to "link" 2d triangle with 3d triangle
	\item gluetags are already calculated and kept in the datastructure for later unfolding
\end{itemize}

\section{Unfolding}

\begin{itemize}
	\item explaining dual graph calculation
	\item explaining algorithm, how simulated annealing is adapted for this problem
	\item explaining calculation of spanning tree
	\item explaining unfolding of 3d triangles
	\item explaining overlap detection how it is done
	\item explaining decision if triangle unfolding was found -> gluetgs are calculated
	\item explaining decision if a gluetag is used or not
	\item gluetag overlap detection similar to triangle overlap -> gluetag consist of two triangles
	\item new energy < old energy we take this step
\end{itemize}

\section{Conflict Detection}

\begin{itemize}
	\item conflict detection most important part
	\item description and example of problematic areas that often arise
	\item triangles checked against other triangles - write how i did it, calculating 
	\item gluetags checked against triangles and other gluetags
\end{itemize}

\chapter{Evaluation}

\section{Performance}

\begin{itemize}
	\item table evaluating the performance against some example meshes
	\item maybe some pictures of unfoldings and original mesh??
	\item maybe graph for performance per polygon increase
\end{itemize}

\section{Limitations}

\begin{itemize}
	\item random walk in annealing can be problematic, especially for big and complex meshes - sometimes finds fast solution, sometimes doesn't
	\item bigger meshes need far more iterations -> more sophisticated algorithm needed to scale better
	\item amount of gluetags cannot be calculated in advance - unfolding decides how many gluetags are needed
\end{itemize}

\chapter{Discussion}

\begin{itemize}
	\item comparison to bruteforcing the unfolding
	\item show some formulas that there are really really many possible unfoldings and trillions of possible gluetag positions and actually that my algorithm works rather well
	\item even sophisticated algorithms for unfolding without gluetags have performance problems if the mesh gets too big
\end{itemize}

\chapter{Conclusion}

\section{Summary}

\begin{itemize}
	\item ???? overall it works fast for smaller meshes, but has its limitations as discussed before
\end{itemize}

\section{Future Work}

\begin{itemize}
	\item post process gluetags to resolve very small overlap areas to cut down computation time
	\item change minimum spanning tree to more sophisticated calculation to get an unfolding faster than with a random walk
	\item find a way to make search space smaller for gluetags to improve performance
\end{itemize}

\backmatter

% Use an optional list of figures.
\listoffigures % Starred version, i.e., \listoffigures*, removes the toc entry.

% Use an optional list of tables.
\cleardoublepage % Start list of tables on the next empty right hand page.
\listoftables % Starred version, i.e., \listoftables*, removes the toc entry.

% Use an optional list of alogrithms.
\listofalgorithms
\addcontentsline{toc}{chapter}{List of Algorithms}

% Add an index.
\printindex

% Add a glossary.
\printglossaries

% Add a bibliography.
\bibliographystyle{alpha}
\bibliography{meshunfolding}

\end{document}